\section{Introduction}
Correlation analysis for all pairs of time series is often the first step of analysis of such data. The notion of correlation allows us to explore the interactional link and dependency relationship between data streams, to discover groups of objects with similar behavior, and, consequently, to detect potential anomalies that a change in correlation may reveal.\newline

In recent years, more and more individuals and companies are collecting data over time. The increasing instrumentation of physical and computing processes has given us unprecedented capabilities to manage massive volumes of data. Time-series exist in almost every field of human activity; they may be information about social and socio-technological processes. For example, In housing and plumbing, there is an analyzing system that can track the amount of fluoride in the drinking water in real time because it may lead to fewer dental cavities some years later. The second example is that almost all online services like YouTube and Apple Music use some recommendation system to find customers with similar shopping patterns and then provide dynamic recommendations based on how a given customer's behavior correlates with others. Data centers have a monitoring system to recognize the servers with correlated performance patterns~\cite{ref16}. And the last one is financial application~\cite{ref15}. The leading stock traders can spot investment opportunities based on the found correlation between currency rates and the stock market. For example, the increase in the exchange rate of the Australian dollar may result in an increase in the New Zealand dollar, which can be delayed by several hours. \newline

The main goal is quantifying these data and studying how they change over time. However, a vast number of devices permanently produce these time-series data. It is not feasible to load these real-time time series data into a stream processing system, which cannot handle the rapidly increasing amount of time-series data. Moreover, users or higher-level applications require immediate responses and cannot afford post-processing. The correlation set discovery problem is NP-hard~\cite{ref11}. So, a concise but powerful model can capture various trend or pattern types of correlation. That is why this discovery is important. \newline

There are different notions of correlation: Pearson correlation coefficient, Spearman correlation coefficient~\cite{ref13}, Kendall correlation coefficient, extended Jaccard coefficient~\cite{ref14}, mutual information~\cite{ref10}, Cosine similarity, Euclidean distance and dynamic time warping~\cite{ref12}. But in this paper, I focus on Pearson correlation coefficients measure. However, some approaches allow us to choose another correlation measure.\newline 

The rest of the paper is organised as follows: In Section \hyperref[sec:2]{2}, I describe the problem of discovery of global correlation for time-series and give a view of 3 approaches to detect it. Section \hyperref[sec:3]{3} overviews local correlation and explains three efficient approaches. Finally, Section \hyperref[sec:4]{4} gives a brief conclusion.